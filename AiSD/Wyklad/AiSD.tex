\documentclass{article}
\usepackage[utf8]{inputenc}
\usepackage[T1]{fontenc}
\usepackage{lmodern}
\usepackage[polish]{babel}
\usepackage{amsmath}
\usepackage{tikz}
\usepackage{algorithm}
\usepackage{algpseudocode}
\usepackage{hyperref}
\usepackage{float}
\usepackage{graphicx}
\usepackage{mathtools}
\usepackage{amsmath}
\usepackage{amsfonts}
\usepackage{amsmath}




\title{Algorytmy i Struktury Danych}
\author{Wojciech Typer}
\date{}

\begin{document}
\maketitle

\begin{algorithm}[H]
\caption{Insertion Sort}\label{alg:insertion_sort}
\begin{algorithmic}[1]
\Procedure{InsertionSort}{A, n}
    \For{$i = 1$ to $n-1$}
        \State $key = A[i]$
        \State $j = i - 1$
        \While{$j \geq 0$ and $A[j] > key$}
            \State $A[j+1] = A[j]$
            \State $j = j - 1$
        \EndWhile
        \State $A[j+1] = key$
    \EndFor
\EndProcedure
\end{algorithmic}
\end{algorithm} 
\vspace{1\baselineskip}
\textbf{Złożoność czasowa:} $O(n^2)$ \par
\textbf{Best case:} w najlepszym przypadku złożoność czasowa będzie wynosić $O(n)$ \par
\textbf{Złożoność pamięciowa:} $O(1)$
\vspace{2\baselineskip}
\begin{figure}[H]
    \centering
    \includegraphics[width=0.8\textwidth]{/home/wojteq18/Pobrane/zdjecia/insert-sort.png}
    \label{fig:example_image}
\end{figure}
\vspace{3\baselineskip}

\begin{algorithm}[H]
    \caption{Merge Sort}\label{alg:merge_sort}
    \begin{algorithmic}[1]
    \Procedure{MergeSort}{A, 1, n}
        \If{|A[1..n]| == 1} 
            \State \Return{A[1..n]}
        \Else
            \State $B = \text{MergeSort}(A, 1, \lfloor n/2 \rfloor)$
            \State $C = \text{MergeSort}(A, \lfloor n/2 \rfloor, n)$
            \State \Return{Merge(B, C)}
        \EndIf
    \EndProcedure
    \end{algorithmic}    
\end{algorithm}
\begin{algorithm}[H]
    \caption{Merge}\label{alg:merge}
    \begin{algorithmic}[1]
    \Procedure{Merge}{X[1..k], Y[1..n]}
        \If{$X = \emptyset$}
            \State \Return{$Y$}
        \ElsIf{$Y = \emptyset$}
            \State \Return{$X$}
        \ElsIf{$X[1] \leq Y[1]$}
            \State \Return{$[X[1]] \times \text{Merge}(X[2..k], Y[1..n])$}   
        \Else
            \State \Return{$[Y[1]] \times \text{Merge}(X[1..k], Y[2..n])$}
        \EndIf
    \EndProcedure
    \end{algorithmic}       
\end{algorithm}
\vspace{1\baselineskip}
\textbf{Złożoność czesowa Merge Sort:} $O(n \log n)$ \par
\textbf{Złożoność pamięciowa Merge Sort:} $O(n)$
\vspace{2\baselineskip}
\begin{figure}[H]
    \centering
    \includegraphics[width=0.5\textwidth]{/home/wojteq18/Pobrane/Merge_sort_algorithm_diagram.svg.png}
    \label{fig:example_image}
\end{figure}
\vspace{2\baselineskip} \par
Istnieje również iteracyjna wersja algorytmu Merge, sort, która została \par
przedstawiona poniżej w postaci pseudokodu.
\begin{algorithm}[H]
    \caption{IterativeMergeSort}\label{alg:iterative_merge}
    \begin{algorithmic}[1]
        \Procedure{IterativeMergeSort}{A[1..n]}
            \For{$size = 1$ \textbf{to} $n-1$ \textbf{by} $size \times 2$}
                \For{$left = 0$ \textbf{to} $n-1$ \textbf{by} $2 \times size$}
                    \State $mid \gets \min(left + size - 1, n - 1)$
                    \State $right \gets \min(left + 2 \times size - 1, n - 1)$
                    \State \Call{Merge}{A, left, mid, right}
                \EndFor
            \EndFor
        \EndProcedure
    \end{algorithmic}
\end{algorithm}
\vspace{1\baselineskip}
\textbf{Złożoność czasowa Iterative Merge Sort:} $O(n \log n)$ - dzieje się tak, \par
ponieważ size jest podwajany o 2 w każdej iteracji, więc potrzebujemy \par
około $ \log_2 n$ iteracji, a w każdej z nich wykonujemy $O(n)$ operacji. \par
\vspace{1\baselineskip}
\textbf{Złożoność pamięciowa Iterative Merge Sort:} $O(n)$ \par
\vspace{1\baselineskip}
\textbf{Notacja asymptotyczna}
O:f(n) = O(g(n)) $\rightarrow (\exists c > 0) (\exists n_0 \in N) : \forall n \geq n_0 \rightarrow 0 \leq f(n) \leq c \cdot g(n)$ \par
\vspace{1\baselineskip}
\begin{figure}[H]
    \centering
    \includegraphics[width=0.5\textwidth]{/home/wojteq18/Pobrane/zdjecia/images.png}
    \label{fig:example_image}
\end{figure}
\vspace{1\baselineskip}
$f(n) = O(g(n)) \rightarrow lim_{n \to \infty} \frac{|f(n)|}{|g(n)|} < \infty$ \par
\vspace{1\baselineskip}
\textbf{Notacja asymptotyczna - własności} \par
\vspace{1\baselineskip}
a) $f(n) = n^3 + O(n^2) \rightarrow (\exists h(n) = O(n^2))(f(n) = n^3 + h(n))$ \par
\vspace{1\baselineskip}
b) $n^2 + O(n) = O(n^2) \rightarrow (\forall f(n) = O(n))(\exists h(n) = O(n^2))(n^2 + f(n)-h(n))$ \par
\vspace{1\baselineskip}
\textbf{Notacja $\Omega$} \par
\vspace{1\baselineskip}
$f(n) = \Omega (g(n)) \rightarrow(\exists c > 0)(\exists n_0 \in N)(\forall n \geq n_0)(c * g(n) \leq |f(n))$ \par
\vspace{1\baselineskip}
\begin{figure}[H]
    \centering
    \includegraphics[width=0.5\textwidth]{/home/wojteq18/Pobrane/zdjecia/1a7d9c3f882e7a237b30f5eb6defa1aa45c6ab22.png}
    \label{fig:example_image}
\end{figure}
\vspace{7\baselineskip}
\textbf{Notacja $\Omega$ - własności} \par
\vspace{1\baselineskip}
a) $n^3 = \Omega (2n^2)$ \par
\vspace{1\baselineskip}
b) $n = \Omega (log(n))$ \par
\vspace{1\baselineskip}
c) $2n^2 = \Omega (n^2)$ \par
\vspace{1\baselineskip}
\textbf{Notacja $\Theta$} \par
\vspace{1\baselineskip}
$f(n) = \Theta (g(n)) \rightarrow (\exists c_1, c_2 > 0)(\exists n_0 \in N)(\forall n  \geq n_0)(c_1 g(n) \leq $ \par $|f(n) \leq c_2 g(n))$ \par
\vspace{1\baselineskip}
\begin{figure}[H]
    \centering
    \includegraphics[width=0.5\textwidth]{/home/wojteq18/Pobrane/zdjecia/theeta.png}
    \label{fig:example_image}
\end{figure}
\vspace{1\baselineskip}
$\Theta (f(n)) = O(f(n)) \cap \Omega (f(n))$ \par
\vspace{1\baselineskip}
\textbf{Notacja o- małe} \par
\vspace{1\baselineskip}
$f(n) = o(g(n)) \rightarrow (\forall c > 0)(\exists n_0 \in N)(\forall n \geq n_0)(|f(n)| < c * |g(n)|)$ \par
\vspace{1\baselineskip}
\textbf{Notacja o- małe - przykłady} \par
\vspace{1\baselineskip}
a) $117n log(n) = o(n^2)$ \par
\vspace{1\baselineskip}
b) $ n^2 = o(n^3)$ \par
\vspace{1\baselineskip}
\textbf{Notacja $\omega$} \par
\vspace{1\baselineskip}
$f(n) = \omega (g(n)) \rightarrow (\forall c > 0)(\exists n_0 \in N)(\forall n \geq n_0)(|f(n)| > c * |g(n)|)$ \par
\vspace{1\baselineskip}
$lim_{n \to \infty} \frac{|f(n)|}{|g(n)|} = \infty$ \par
\vspace{1\baselineskip}
\begin{figure}[H]
    \centering
    \includegraphics[width=0.5\textwidth]{/home/wojteq18/Pobrane/zdjecia/allnotations.png}
    \label{fig:example_image}
\end{figure}
\vspace{1\baselineskip}
\textbf{Rekurencje} \par
\vspace{1\baselineskip}
Metoda podstawiania (metoda dowodzenia indukcyjnego) \par
    \hspace{20pt}1. Zgadnij odpowiedź (bez stałych) \par
    \hspace{20pt}2. Sprawdź przez indukcję, czy dobrze zgadliśmy \par
    \hspace{20pt}3. Znajdź stałe \par
\vspace{1\baselineskip}
Przykład 1: \par
\vspace{1\baselineskip}
$T(n) = 4T(\frac{n}{2}) + n$   \par
Pierwszy strzał: $T(n) = O(n^3)$ \par
Cel: pokazać, że $(\exists c > 0) T(n) \leq c * n^3$ \par
Krok początkowy: $T(1) = \Theta (1) = c * 1^3 = c$ \par
Krok indukcyjny: zał. że, $(\forall_(k < n)) (T(k) \leq c * k^3) = $ \par
Dowód: $T(n) = 4T(\frac{n}{2}) + n \leq 4c * (\frac{n}{2})^3 + n = \frac{1}{2}cn^3 + n =$ \par
$= cn^3 - \frac{1}{2}cn^3 + n = cn^3 - (\frac{1}{2}cn^3 - n) \leq cn^3$ \par
Pokazaliśmy, że $T(n) = O(n^2)$ \par
\vspace{1\baselineskip}
Spróbujmy wzmocnić zał. indukcyjne: $T(n) \leq c_1 n^2 - c_2 n$ \par
$T(n) \leq 4T(\frac{n}{2}) + n \leq 4(c_1 (\frac{n}{2})^2 - c_2 \frac{n}{2}) + n = $ \par
$ = c_1 n^2 - 2c_2 n + n = c_1 n^2 - (2c_2 - 1)n \leq c_1 n^2 - c_2 n$ \par
Musimy dobrać takie $c_1 i c _2$, aby $2c_1 \geq c_2$ \par
Wówczas otrzymamy $T(1) = O(1) \leq c_1 1^2 - c_2 1$ \par
\vspace{9\baselineskip}
Przykład 2: \par
\vspace{1\baselineskip}
$T(n) = 2T(\sqrt{n}) + log(n)$ \par
Załóżmy, że n jest potęgą dwójki $n = 2^m \rightarrow m = log(n)$ \par
$T(2^m) = 2T(2^{m/2}) + m$ \par
oznaczmy $T(2^m) = S(m)$ \par
$T(2^m) = 2T(2^{m/2}) + m \rightarrow 2S(m/2) + m$ \par
$S(m) = O(m log(m))$ \par
$T(n) = O(log(n) log(log(n)))$ (formalnie powinniśmy to udowodnić) \par  
\vspace{1\baselineskip}
\textbf{Drzewo rekursji} \par
\vspace{1\baselineskip}
Przykład : $T(n) = T(\frac{n}{2}) +T (\frac{n}{4}) + n^2$ \par
\vspace{1\baselineskip}
\begin{center}
    \begin{tikzpicture}
        [level distance=1.5cm,
        level 1/.style={sibling distance=4cm},
        level 2/.style={sibling distance=2cm}]
        \node {$n^2$}
            child {node {$\frac{n^2}{4}$}
                child {node {$\frac{n^2}{16}$}}
                child {node {$\frac{n^2}{64}$}}
            }
            child {node {$\frac{n^2}{16}$}
                child {node {$\frac{n^2}{64}$}}
                child {node {$\frac{n^2}{256}$}}
            };
          \node[draw=none] at (-4.5,0) {$n^2$};
          \node[draw=none] at (-4.5,-1.5) {$\frac{5}{16}n^2$};
          \node[draw=none] at (-4.5,-3) {$\frac{25}{256}n^2$};
    \end{tikzpicture}
\end{center}
\vspace{1\baselineskip}
Trzeba pamiętać, że drzewo rekursji samo w sobie nie jest formalnym rozwiązaniem problemu. Nie można go urzywać do dowodzenia złożoności algorytmów.
Jest to jedynie intuicyjne podejście do problemu. Formmalnie T(n) należałoby policzyć jako sumę wszystkich wierzchołków w drzewie rekursji:
\[
    T(n) = \sum^{\infty}_{k=0} \left(\frac{5}{16}\right)^k \cdot n^2 = n^2 \sum^{\infty}_{k=0} \left(\frac{5}{16}\right)^k = n^2 \frac{1}{1-\frac{5}{16}} = n^2 \frac{16}{11} = \frac{16}{11}n^2
\]
Widzimy zatem, że $T(n) = O(n^2)$ \par
\vspace{1\baselineskip}
\textbf{Master Theorem} \par
\vspace{1\baselineskip}
Niech $a \geq 1, b > 1, f(n), d \in N$ oraz $f(n)$ będzie funkcją nieujemną. Rozważmy rekurencję: \par
\[
    T(n) = aT(\frac{a}{b}) + \Theta(n^d)
\] 
Wówczas: \par
\begin{itemize}
    \item $\Theta(n^d)$, jeśli $d > log_b a$
    \item $\Theta (n^d log(n))$, jeśli $d = log_b a$
    \item $\Theta(n^{log_b a})$, jeśli $d < log_b a$
\end{itemize}
Do przedstawienia problemu użyjemy drzewa rekursji. Rozważmy rekurencję:
\[
    T(n) = aT(\frac{n}{b}) + \Theta(n^d)
\]
\begin{center}
\begin{tikzpicture}
    [level distance=1.5cm,
    level 1/.style={sibling distance=4cm},
    level 2/.style={sibling distance=2cm}]
    \node {$c \cdot n^d$}
        child {node {$c \cdot \left(\frac{n}{b}\right)^d$}
            child {node {$c \cdot \left(\frac{n}{b^2}\right)^d$}}
            child {node {$c \cdot \left(\frac{n}{b^2}\right)^d$}}
        }
        child {node {$c \cdot \left(\frac{n}{b}\right)^d$}
            child {node {$c \cdot \left(\frac{n}{b^2}\right)^d$}}
            child {node {$c \cdot \left(\frac{n}{b^2}\right)^d$}}
        };
      \node[draw=none] at (-4.5,0) {$n^d$};
      \node[draw=none] at (-4.5,-1.5) {$\frac{n^d}{b^d}$};
      \node[draw=none] at (-4.5,-3) {$\frac{n^d}{b^{2d}}$};
  \end{tikzpicture} 
\end{center}
\begin{enumerate}
    \item suma kosztoów w $k$--tym kroku
        \[
            a^k c (\frac{n}{b^k})^d = c (\frac{a}{b^d})^k n^d
        \]
        gdzie $c(\frac{n}{b^k})^d$ to koszt jednego podproblemu w $k$--tym kroku
    \item obliczenie wysokości drzewa:
        \[
            \frac{n}{b^h} = 1 \rightarrow h = \log_b n
        \]
    \item Obliczenie $T(n)$
    \begin{align*}
        T(n) &= \Theta\left(\sum^{\log_b n}_{k=0} c\frac{a}{b^k}n^d\right) \\
             &= \Theta\left(c \cdot n^d \sum^{\log_b n}_{k=0} \left(\frac{a}{b^d}\right)^k\right) \\
             &= \Theta\left(c \cdot n^d \frac{1-\left(\frac{a}{b^d}\right)^{\log_b n + 1}}{1-\frac{a}{b^d}}\right) \\
             &\implies T(n) = \Theta(n^d)
    \end{align*}
    
    \item rozważmy 3 przypadki:
        \begin{enumerate}
            \item $d > \log_b a$ 
                \[
                    T(n) = \Theta(n^d)
                \]
            \item $d = \log_b a$ 
                \[
                    T(n) = \Theta(n^d \log n)
                \]
            \item $d < \log_b a$
                \[
                    T(n) = \Theta(n^{\log_b a})
                \]
        \end{enumerate}
\end{enumerate}

\subsubsection*{Przykłady}
\begin{itemize}
    \item $T(n) = 4T(\frac{n}{2}) + 11n$ \newline
        Wtedy kożystając z \textbf{Master Theorem} mamy:
        \[
            a = 4, b = 2, d = 1
        \]
        Jak i również
        \[
            \log_b a = \log_2 4 = 2 > 1 = d \implies T(n) = \Theta(n^2)
        \]
    \item $T(n) = 4T(\frac{n}{3}) + 3n^2$ \newline
        Wtedy
        \[
            a = 4, b = 3, d = 2
        \]
        Jak i również
        \[
            \log_b a = \log_3 4 < 2 = d \implies T(n) = \Theta(n^2)
        \]
    \item $T(n) = 27T(\frac{n}{3}) + \frac{n^2}{3}$ \newline
        Wtedy
        \[
            a = 27, b = 3, d = 2
        \]
        Jak i również
        \[
            \log_b a = \log_3 27 = 3 > 2 = d \implies T(n) = \Theta(n^3\log n)
        \]
\end{itemize}

\subsection*{Metoda dziel i zwyciężaj (D\&C)}
Na czym ona polega?
\begin{enumerate}
    \item Podział problemu na mniejsze podproblemy 
    \item Rozwiazanie rekurencyjnie mniejsze podpoblemy
    \item połącz rozwiązania podproblemów w celu rozwiązania problemu wejściowego
\end{enumerate}
\subsubsection*{Algorytm -- Binary Search}
\begin{itemize}
    \item \textbf{Input}: posortowania tablica \texttt{A[1..n]} oraz element \texttt{x}
    \item \textbf{Output}: indeks \texttt{i} taki, że \texttt{A[i] = x} lub \texttt{0} jeśli \texttt{x} nie występuje w \texttt{A}
        \item przebieg algorytmu: 
            \begin{algorithm}[H]
                \caption{Binary Search}
                \begin{algorithmic}[1]
                    \Procedure{BinarySearch}{A, x}
                        \State $l = 1$
                        \State $r = |A|$
                        \While{$l \leq r$}
                            \State $m = \lfloor \frac{l+r}{2} \rfloor$
                            \If{$A[m] = x$}
                                \State \Return{$m$}
                            \ElsIf{$A[m] < x$}
                                \State $l = m + 1$
                            \Else
                                \State $r = m - 1$
                            \EndIf
                        \EndWhile
                        \State \Return{0}
                    \EndProcedure
                \end{algorithmic}
            \end{algorithm}
        \item \textbf{Asypmtotyka}
            Algorytm spełnia następująca rekurencje:
            \[
                T(n) = T(\frac{n}{2}) + \Theta(1)
            \]
            Rozwiązując za pomocą \textbf{Master Theorem} otrzymujemy:
            \[
                T(n) = \Theta(\log n)
            \]
\end{itemize}
\vspace{1\baselineskip}
\subsubsection*{Divide \& Conquer}

\textbf{Problem:} Obliczenie $x^n$.  

Rozwiązanie naiwną metodą iteracyjną:  
\[
x^n = x \cdot x \cdot \dots \cdot x \quad \Rightarrow \quad \Theta(n)
\]

Rozwiązanie za pomocą Divide \& Conquer:  

\[
x^n =
\begin{cases}
    (x^{\frac{n}{2}}) \cdot (x^{\frac{n}{2}}), & \text{gdy } n \text{ jest parzyste} \\
    (x^{\frac{n-1}{2}}) \cdot (x^{\frac{n-1}{2}}) \cdot x, & \text{gdy } n \text{ jest nieparzyste}
\end{cases}
\]

Rekurencyjna złożoność czasowa:
\[
T(n) = T(n/2) + \Theta(1) = \Theta(\log n)
\]

---

\textbf{Problem:} Obliczenie $n$-tej liczby Fibonacciego  

Metoda rekurencyjna:
\[
F(n) = F(n-1) + F(n-2)
\]
Ma ona złożoność wykładniczą:
\[
\Theta (\phi^n), \quad \text{gdzie } \phi = \frac{1 + \sqrt{5}}{2}
\]

Drzewo rekurencyjne dla $F_4$:

\begin{center}
\begin{tikzpicture}[level distance=1.5cm,
    level 1/.style={sibling distance=6cm},
    level 2/.style={sibling distance=3cm},
    level 3/.style={sibling distance=1.5cm}]
    
  \node {$F_4$}
      child {node {$F_3$}
          child {node {$F_2$}
              child {node {$F_1$}}
              child {node {$F_0$}}
          }
          child {node {$F_1$}}
      }
      child {node {$F_2$}
          child {node {$F_1$}}
          child {node {$F_0$}}
      };

\end{tikzpicture}
\end{center}

\textbf{Wzór jawny:}
\[
F_n = \frac{1}{\sqrt{5}} \left( \phi^n - (-\phi)^{-n} \right)
\]


Obliczanie $F_n$ macierzą:
Zamiast rekurencji można użyć potęgowania macierzy, co daje optymalną złożoność.  
Dla każdego $n \geq 0$ zachodzi:

\[
\begin{bmatrix}
    1 & 1 \\
    1 & 0
\end{bmatrix}^n
=
\begin{bmatrix}
    F_{n+1} & F_n \\
    F_n & F_{n-1}
\end{bmatrix}
\]

Potęgowanie macierzy metodą szybkiego potęgowania daje czas:
\[
\Theta(\log n)
\]
co jest znaczną poprawą w porównaniu do wykładniczej rekurencji.

\textbf{Mnożenie liczb binarnych metodą Divide \& Conquer}

\textbf{Wejście:} $x, y$  
\textbf{Wyjście:} $x \cdot y$

Każdą liczbę można rozbić na dwie połowy:
\[
x = x_L \cdot 2^{\frac{n}{2}} + x_R
\]
\[
y = y_L \cdot 2^{\frac{n}{2}} + y_R
\]

Podstawiając do iloczynu:
\[
xy = (x_L \cdot 2^{\frac{n}{2}} + x_R) \cdot (y_L \cdot 2^{\frac{n}{2}} + y_R)
\]

Po rozwinięciu:
\[
xy = x_L y_L \cdot 2^n + (x_L y_R + x_R y_L) \cdot 2^{\frac{n}{2}} + x_R y_R
\]

Rekurencyjna zależność czasowa:
\[
T(n) = 4T(n/2) + \Theta(n)
\]

Zastosowanie \textbf{Master Theorem} daje:
\[
T(n) = \Theta(n^2)
\]
co pokazuje, że metoda ta nie poprawia złożoności względem standardowego mnożenia. 

\vspace{1\baselineskip}
\textbf{Optymalizacja: metoda Gaussa}

Zamiast wykonywać 4 mnożenia rekursywne, można zastosować zasadę Gaussa:
\[
xy = x_L y_L \cdot 2^n + ((x_L + x_R)(y_L + y_R) - x_L y_L - x_R y_R) \cdot 2^{\frac{n}{2}} + x_R y_R
\]

Dzięki temu zamiast 4 mnożeń wykonujemy tylko 3:
\[
T(n) = 3T(\frac{n}{2}) + \Theta(n)
\]

Zastosowanie \textbf{Master Theorem} daje:
\[
T(n) = \Theta(n^{\log_2 3})
\]

\begin{algorithm}[H]
    \caption{Multiply - Mnożenie dużych liczb binarnych metodą Gaussa}
    \label{alg:multiply}
    \begin{algorithmic}[1]
        \Procedure{multiply}{x, y}
            \State $n \gets \max(|x|, |y|)$ 
            \If{$n = 1$} 
                \State \Return{$x \cdot y$}
            \EndIf
            \State $m \gets \lceil {n/2} \rceil$
            \State $x_L, x_R \gets$ 
            \State $y_L, y_R \gets$ 
            \State $p_1 \gets \Call{multiply}{x_L, y_L}$
            \State $p_2 \gets \Call{multiply}{x_R, y_R}$
            \State $p_3 \gets \Call{multiply}{(x_L + x_R), (y_L + y_R)}$
            \State \Return{$p_1 \cdot 2^{2m} + (p_3 - p_1 - p_2) \cdot 2^m + p_2$}
        \EndProcedure
    \end{algorithmic}
\end{algorithm}
\textbf{QuickSort}
\begin{algorithm}[H]
    \caption{QuickSort - Sortowanie szybkie}
    \label{alg:quicksort}
    \begin{algorithmic}[1]
        \Procedure{quicksort}{A, low, high}
            \If{$low < high$}
                \State $p \gets \Call{partition}{A, low, high}$
                \State \Call{quicksort}{A, low, p - 1}
                \State \Call{quicksort}{A, p + 1, high}
            \EndIf
        \EndProcedure
        \\
        \Procedure{partition}{A, low, high}
            \State $pivot \gets A[high]$
            \State $i \gets low - 1$
            \For{$j \gets low$ \textbf{to} $high - 1$}
                \If{$A[j] \leq pivot$}
                    \State $i \gets i + 1$
                    \State \Call{swap}{A[i], A[j]}
                \EndIf
            \EndFor
            \State \Call{swap}{A[i + 1], A[high]}
            \State \Return{$i + 1$}
        \EndProcedure
    \end{algorithmic}
\end{algorithm}
\begin{figure}[H]
    \centering
    \includegraphics[width=0.7\textwidth]{/home/wojteq18/Pobrane/zdjecia/quick.png}
    \label{fig:example_image}
\end{figure}

\begin{algorithm}[H]
    \caption{Hoare Partition}
    \label{alg:quicksort}
    \begin{algorithmic}[1]
        \Procedure{hoare\_partition}{A, p, q}
            \State $pivot \gets A\left[\left\lfloor \frac{p + q}{2} \right\rfloor\right]$
            \State $i \gets p - 1$
            \State $j \gets q + 1$
            \While{true}
                \Repeat
                    \State $i \gets i + 1$
                \Until{$A[i] \geq pivot$}
                \Repeat
                    \State $j \gets j - 1$
                \Until{$A[j] \leq pivot$}
                \If{$i \geq j$}
                    \State \Return $j$
                \EndIf
                \State swap($A[i], A[j]$)
            \EndWhile
        \EndProcedure
    \end{algorithmic}
\end{algorithm}

\par\textbf{Analiza worst-case QuickSorta}

\par
$T(n) = T(n - 1) + T(0) + \Theta(n) = T(n - 1) + \Theta(n)$

\par
Drzewo rekurencji (dla przypadku pesymistycznego, tj. jednostronny podział):

\begin{center}
\begin{tikzpicture}[
    level distance=1.4cm,
    every node/.style={circle,draw},
    level 1/.style={sibling distance=5cm},
    level 2/.style={sibling distance=3cm},
    level 3/.style={sibling distance=2cm},
    level 4/.style={sibling distance=1.5cm}
    ]
\node {$c_n$}
  child {node {$c_{n-1}$}
    child {node {$c_{n-2}$}
      child {node {$c_{n-3}$}
        child {node {$\cdots$}
          child {node {$c_1$}
            child[missing]
            child[missing]
          }
          child[missing]
        }
        child[missing]
      }
      child[missing]
    }
    child[missing]
  }
  child {node {$\Theta(1)$}}; % prawa strona (najmniejsza partycja)
\end{tikzpicture}
\end{center}


\par
$T(n) \leq \sum_{i=1}^{n} c \cdot i = c \cdot \sum_{i=1}^{n} i = \Theta(n^2)$
\vspace{1\baselineskip}

\par\textbf{Analiza best-case}

\par
Jeśli pivot zawsze dzieli tablicę na dwie równe części:

\[
T(n) = 2T\left(\frac{n}{2}\right) + \Theta(n)
\Rightarrow T(n) = \Theta(n \log n)
\]

\par\textbf{Analiza average-case}

\par
Niech $T_n$ oznacza liczbę porównań dla tablicy długości $n$.

\[
x_k =
\begin{cases}
    1, & \text{jeśli partition dzieli tablicę na } (k,\; n - k - 1) \\[0.5em]
    0, & \text{w przeciwnym wypadku}
\end{cases}
\]


\[
T_n = \sum_{k=0}^{n-1} x_k \cdot (T_k + T_{n-k-1}) + (n - 1)
\]

\par
Liczymy wartość oczekiwaną:

\[
E(T_n) = \sum_{k=0}^{n-1} \mathbb{E}(x_k) \cdot \left( \mathbb{E}(T_k) + \mathbb{E}(T_{n-k-1}) \right) + (n - 1)
\]

\[
\mathbb{E}(x_k) = \frac{1}{n} \quad \text{(bo pivot jest losowy)}
\]

\[
E(T_n) = \frac{1}{n} \sum_{k=0}^{n-1} \left( E(T_k) + E(T_{n-k-1}) \right) + (n - 1)
\]

\[
= \frac{2}{n} \sum_{k=0}^{n-1} E(T_k) + (n - 1)
\]

\[
\Rightarrow E(T_n) = \Theta(n \log n)
\]

    \vspace{1\baselineskip}
    \textbf{Analiza avg Case'a}
    $T_n \rightarrow$ Liczba porównań elementów sortowanej tablicy: |A| = n \newline

\[
x_k =
\begin{cases}
    1, & \mbox{jeśli partition dzieli tablicę na } (k,\, n - k - 1) \\[0.6em]
    0, & \mbox{w przeciwnym wypadku}
\end{cases}
\]


\[
    T_n =
    \begin{cases}
        T_0 + T_{n-1} + n-1, gdy (0, n-1) -split \\
        T_1 + T_{n-2} + n -1, gdy (1, n-2) -split \\
        ... \\
        T_k + T_{n-1-k} + n - 1, gdy (k, n-k-1) -split \\
        ... \\
        T_{n-1} + T_0 + n - 1, gdy (n-1, o) -split
    \end{cases}
    \]


    $T_n = \sum_{k=0}^{n-1} x_k (T_k + T_{n-k-1}) + n - 1$ \newline
    
    liczymy wartosć oczekiwaną: \newline

    $E(T_n) = E (\sum{k=0}^{n-1} X_k (T_k + T_{n-k-1} + n - 1))$ \newline
    $E(T_n) = \sum_{k=0}^{n-1} E(X_k \cdot (T_k + T_{n-k-1}) + n - 1) $ \newline
    $E(T_n) = \sum_{k=0}^{n-1} E(X_k) - E(T_k + T_{n-k-1} - n - 1)$ \newline
    $E(T_n) = \frac{1}{n} \cdot \sum_{k=0}^{n-1} E(T_k) + \sum (E(T_{n-k-1}))$ \newline


    \textbf{Dual pivot quicksort} \newline

    \begin{figure}[H]
        \centering
        \includegraphics[width=0.7\textwidth]{/home/wojteq18/Pobrane/zdjecia/dpq.png}
        \label{fig:example_image}
    \end{figure}

    \textbf{Wartość oczekiwana: } 

    $E(\mbox{liczba porównań w dual pivot partition}) \approx \frac{16}{9}n$ 

    $E(\mbox{liczba porównań w dual pivot qs sedwick}) \approx \frac{32}{15}n logn$  \newline


    \textbf{Yaroslavsky dual pivot qs} 

    $E(\mbox{liczba porównań w partition}) \approx \frac{19}{12}n$ 

    $E(\mbox{liczba porównań w Dual Pivot qs Yaroslavsky}) \approx 1.9 n logn$ \newline

    \textbf{Strategia count}

    $E(\mbox{liczba porównań w Count Partition}) \approx \frac{3}{2}n$

    $E(\mbox{liczba porównań w Dual Pivot qs z count}) \approx 1.8 n logn$ \newline

    \textbf{Comparsion Model} 

    Dolne ograniczenie na liczbę porównań w problemie sortowania \par
    w Comparsion Modelwynosi $\Omega(n logn)$ \newline

    D-d: \par
    \begin{itemize}
        \item dla dowolnego algorytmu sortującego możemy znależć odpowiadające mu drzewo decyzyjne
        \item n! liści w binarnym drzewie decyzyjnym
        \item drzewo binarne pełne o wysokości h ma co najmniej $2^h$ liści
        \item ale liści w drzewie decyzyjnym powinno być co najmniej n!, zatem: \par
        $2^h \leq n!$ / lg \par
        $h \leq \log_{2}n!$ \par
        $lg n! = lg(\sqrt{s \pi n} (\frac{n}{e})^n (1 + o(1)))$ \par
        $lg (\frac{n}{e})^n + lg (\sqrt(2 \pi n)(1 + o(1)))$ \par
        $n logn - n lg e + lg(\sqrt{2 \pi n} (2 + o(1))) = \Omega(n logn)$
    \end{itemize}    \par

    Sortowanie: \par
    Input: $|a| = n, \forall i \in \{1, ..., k\}$ \par
    Output: posortowana rosnąco tablica A \par

    \begin{algorithm}[H]
        \caption{CountingSort}
        \label{alg:countingsort}
        \begin{algorithmic}[1]
            \Procedure{counting\_sort}{A, n, k}
                \For{$i = 1$ \textbf{to} $k$}
                    \State $C[i] \gets 0$
                \EndFor
                \For{$i = 1$ \textbf{to} $n$}
                    \State $C[A[i]] \gets C[A[i]] + 1$
                \EndFor
                \For{$i = 2$ \textbf{to} $k$}
                    \State $C[i] \gets C[i] + C[i - 1]$
                \EndFor
                \For{$i = n$ \textbf{downto} $1$}
                    \State $B[C[A[i]]] \gets A[i]$
                    \State $C[A[i]] \gets C[A[i]] - 1$
                \EndFor
                \State \Return $B$
            \EndProcedure
        \end{algorithmic}
    \end{algorithm} \par

    \begin{figure}[H]
        \centering
        \includegraphics[width=0.7\textwidth]{/home/wojteq18/Pobrane/zdjecia/cs.png}
        \label{fig:example_image}
    \end{figure} \par

    Złożoność obliczeniowa Counting Sorta: \par
    $\Theta (n + k)$ gdzie $k = O(n)$ \par

    \vspace{1\baselineskip}

    \textbf{Stable Sorting Property} \par

    Algorytm zachowuje kolejność równych sobie elementów z tablicy wejściowej

    \vspace{5\baselineskip}

    \textbf{RadixSort} \par

    \begin{algorithm}[H]
        \caption{RadixSort}
        \label{alg:radixsort}
        \begin{algorithmic}[1]
            \Procedure{radix\_sort}{A, n, d}
                \For{$i = 1$ \textbf{to} $d$}
                    \State $counting\_sort(A, n, 9)$
                \EndFor
                \State \Return $A$
            \EndProcedure
        \end{algorithmic}
    \end{algorithm} \par
    \vspace{1\baselineskip}

    \textbf{Złożoność obliczeniowa RadixSorta} \par
    \begin{itemize}
        \item n liczb b'bitowych
        \item liczb b bitowych dzielimy na r-bitowe cyfry
        \item cyfry są z |{$0,...,2^n-1$}| = $2^n$ \par
        \item Counting Sort sortujący n liczb względem jednej cyfry
    \end{itemize} \par
    Zatem RadixSort będzie miał złożoność obliczneiową: \par
    $\Theta(\frac{b}{r} \cdot (n + 2^r))$ \par
    Co po wykonaniu skomplikowanej analizy daje: \par
    $\Theta(d \cdot n)$ \par

    \vspace{2\baselineskip}
    \textbf{Statystyki pozycyjne} \par
    \vspace{1\baselineskip} \par
    \textbf{Def: } k-tą statystyką pozycyjną nazywam← k-tą najmniejszą wartość z \par
    zadanego zbioru \par
    przykład: \par
    \begin{itemize}
        \item $k=1 \rightarrow O(n)$
        \item $k=n$ $\rightarrow O(n)$
        \item $k=\lfloor \rightarrow \text { sortowanie } O(n \log n)$
    \end{itemize}
    \vspace{1\baselineskip}
    \begin{algorithm}[H]
        \caption{RandomSelect}
        \label{alg:randomselect}
        \begin{algorithmic}[1]
            \Procedure{random\_select}{A, p, q, i}
                \If{$p = q$}
                    \State \Return $A[p]$
                \EndIf
                \State $r \gets$ RandPartition($A$, $p$, $q$)
                \State $k \gets r - p + 1$
                \If{$i = k$}
                    \State \Return $A[r]$
                \ElsIf{$i < k$}   
                    \State \Return \Call{random\_select}{$A$, $p$, $r - 1$, $i$}
                \Else
                    \State \Return \Call{random\_select}{$A$, $r + 1$, $q$, $i - k$}
                \EndIf         
            \EndProcedure
        \end{algorithmic}
    \end{algorithm}
    
    \vspace{1\baselineskip}
    \textbf{Select algorithm} \par
    \begin{itemize}
        \item dzielimy A[p..q] na $\frac{n}{\lfloor 5 \rfloor}$ pięcioelementowych częsci
        oraz ostanią część na $\leq$ 5 elementów
        \item Sortujemy te grupy i wybieramy z każdej z nich medianę
        \item Znajdujemy medianę M. Select(M, 1, $\frac{n}{5}$, $\frac{n}{10}$)
        \item Ustalamy X jako pivot; Partition(A, p, q) i tak samo jak w RandomSelect
    \end{itemize}

    \textbf{Select} \par
    Select(A, K) $\rightarrow$ T(n) \par
    \begin{itemize}
        \item Dziel na 5 elementowe tablice i znajdź ich medianę $\rightarrow \Theta(n)$
        \item Select (...) $\rightarrow$ znajdź medianę median $\rightarrow T(\lceil{\frac{n}{5}}\lceil)$ \par
        \item Użyj mediany median jako pivot w Partition $\rightarrow \Theta(n)$
        \item Idź do lewej albo prawej podtablicy w zależności od indeksu pivota i szukaj statystyki pozycyjnej
    \end{itemize}
    Otrzymujemy: $t(n) = T(\lceil{\frac{n}{5}}\lceil) + \Theta(?)$ \par
    \vspace{9\baselineskip}
    \textbf{Struktury danych} \par
    Set interface: \par
    \begin{itemize}
        \item build (A) - buduje set z danych zawartych w A
        \item length - zwraca moc zbioru
        \item find (k) - zwraca element zbioru o kluczu równym k
        \item insert (k) - dodaje element o kluczu k do zbioru
        \item delete (k) - usuwa element o kluczu k ze zbioru
        \item find\_max - zwróc element o największym kluczu
        \item find\_min - zwróć element o najmniejszym kluczu
        \item find\_prev - zwraca element poprzedni od klucza
    \end{itemize}
    \vspace{1\baselineskip}
    \textbf{Binary Search Tree} \par
    BST property : \par
    \begin{itemize}
        \item x $\in$ T - x jest węzłem drzewa T
        \item Wówczas każdy y $in$ x.left ma y.key < x.key
        \item key y $in$ x.right ma y.key > x.key
    \end{itemize}
    \vspace{1\baselineskip}
    \textbf{Inorder Tree Walk} \par
    \begin{algorithm}[H]
        \caption{Inorder Tree Walk}\label{alg:inorder_tree_walk}
        \begin{algorithmic}[1]
        \Procedure{InorderTreeWalk}{x $\in$ T}
            \If{$x \neq \text{null}$}
                \State \Call{InorderTreeWalk}{x.left}
                \State print(x)
                \State \Call{InorderTreeWalk}{x.right}
            \EndIf
        \EndProcedure
        \end{algorithmic}
    \end{algorithm}
    \newpage
    \textbf{Tree Search} \par
    \begin{algorithm}[H]
        \caption{TreeSearch}\label{alg:tree_search}
        \begin{algorithmic}[1]
        \Procedure{TreeSearch}{x $\in$ T, k}
            \If{$x = \text{null} \lor k = x.key$}
                \State \Return x
            \ElsIf{$k < x.key$}
                \State \Return \Call{TreeSearch}{x.left, k}
            \Else
                \State \Return \Call{TreeSearch}{x.right, k}
            \EndIf
        \EndProcedure
        \end{algorithmic}
    \end{algorithm}
    

    
\end{document} 