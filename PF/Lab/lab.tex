\documentclass[11pt,a4paper]{article}

% Kodowanie i obsługa języka polskiego
\usepackage[utf8]{inputenc}      % Kodowanie wejścia
\usepackage[T1]{fontenc}         % Kodowanie fontów
\usepackage[polish]{babel}       % Obsługa języka polskiego

% Ustawienia marginesów i odstępów
\usepackage{geometry}
\geometry{margin=2.5cm}
\usepackage{parskip}  % Czytelniejsze akapity (bez wcięć)

% Pakiety matematyczne i inne przydatne
\usepackage{amsmath,amssymb,amsthm}  % Pakiety do matematyki
\usepackage{graphicx}                % Obsługa grafiki
\usepackage{hyperref}                % Linki i spis treści
\usepackage{algorithm}               % Algorytmy
\usepackage{algpseudocode}           % Pseudokod
\usepackage{fancyvrb}                % boxy wokol sekcji
\usepackage{listings}

\title{Programowanie funkcyjne - laboratoria}
\author{Wojciech Typer}
\date{}

\begin{document}
\maketitle

\section*{zadanie 1}
power x y = power $y^x$

p2 = power 4 $\rightarrow$ power 4 y = $y^4$

p3 = power 3

(p2 . p3) 2 = p2(p3 2) = p2 8 = $8^4$ = 4096

p2 :: Int -> Int

p3 :: Int -> Int

(p2 . p3) :: Int -> Int

Wyrażenia lambda:

power = $\lambda x \rightarrow \lambda y \rightarrow y ^ x$

p2 = $\lambda y \rightarrow y ^ 4$

p3 = $\lambda y \rightarrow y ^ 3$

\bigskip

\section*{zadanie 4}
plus = $\lambda x y \rightarrow x + y$

multi = $\lambda x y \rightarrow x * y$

\bigskip

\section*{zadanie 5}
\textbf{haskell}:

$\lambda x \rightarrow 1 + x * (x + 1)$

\textbf{python}:

f = lambda x: 1 + x * (x + 1)

\bigskip

\section*{zadanie 6}
Ustalmy zbiory \( A, B, C \). Niech
\[
\text{curry} : C^{B \times A} \to (C^B)^A
\]
będzie funkcją zadaną wzorem:

\[
\text{curry}(\varphi) = \lambda a \in A \to (\lambda b \in B \to \varphi(b, a)).
\]

oraz niech

\[
\text{uncurry} : (C^B)^A \to C^{B \times A}
\]

będzie zadana wzorem:

\[
\text{uncurry}(\psi)(b, a) = (\psi(a))(b).
\]

\begin{enumerate}
    \item Pokaż, że \( \text{curry} \circ \text{uncurry} = \text{id}_{(C^B)^A} \) oraz \( \text{uncurry} \circ \text{curry} = \text{id}_{C^{B \times A}} \).

    \item Wywnioskuj z tego, że \( |(C^B)^A| = |C^{B \times A}| \). Przypomnij sobie dowód tego twierdzenia, który poznałeś na pierwszym semestrze studiów.

    \item Spróbuj zdefiniować w języku Haskell odpowiedniki funkcji \texttt{curry} i \texttt{uncurry}.
\end{enumerate}

\bigskip
\hrule
\bigskip

\begin{enumerate}
    \item Pokażemy, że \( \text{curry} \circ \text{uncurry} = \text{id}_{(C^B)^A} \) oraz \( \text{uncurry} \circ \text{curry} = \text{id}_{C^{B \times A}} \).
    \begin{itemize}
        \item \( \text{curry} \circ \text{uncurry} \)
        \[
            (\text{curry} \circ \text{uncurry})(\psi)
            = \text{curry}(\text{uncurry}(\psi))
            = \text{curry}(\lambda a \in A \to (\lambda b \in B \to \psi(a)(b)))
            = \lambda a \in A \to (\lambda b \in B \to \psi(a)(b)).
        \]
        \item \( \text{uncurry} \circ \text{curry} \)
        \[
            (\text{uncurry} \circ \text{curry})(\varphi)
            = \text{uncurry}(\text{curry}(\varphi))
            = \text{uncurry}(\lambda a \in A \to (\lambda b \in B \to \varphi(b, a)))
            = \lambda b \in B \to (\lambda a \in A \to \varphi(b, a)).
        \]
    \end{itemize}

    \item Możemy pokazać że \texttt{curry} i \texttt{uncurry} są iniekcjami niewprost, nakładając odpowiednio przeciwne funkcje na obie strony równości:
    \begin{itemize}
        \item Załóżmy, że \( \text{curry}(\varphi_1) = \text{curry}(\varphi_2) \). Wtedy:
        \[
            \text{curry}(\varphi_1)(a)(b) = \text{curry}(\varphi_2)(a)(b)
            \quad\Rightarrow\quad
            \varphi_1(b, a) = \varphi_2(b, a)
            \quad\Rightarrow\quad
            \varphi_1 = \varphi_2.
        \]
        \item Załóżmy, że \( \text{uncurry}(\psi_1) = \text{uncurry}(\psi_2) \). Wtedy:
        \[
            \text{uncurry}(\psi_1)(b, a) = \text{uncurry}(\psi_2)(b, a)
            \quad\Rightarrow\quad
            \psi_1(a)(b) = \psi_2(a)(b)
            \quad\Rightarrow\quad
            \psi_1 = \psi_2.
        \]
    \end{itemize}
    A więc istnieje biekcja między \( (C^B)^A \) i \( C^{B \times A} \), co oznacza, że te zbiory mają taką samą moc.

    \item W języku Haskell funkcje \texttt{curry} i \texttt{uncurry} można zdefiniować następująco:
\begin{Verbatim}[frame=single]
curry :: ((b, a) -> c) -> a -> b -> c
curry f x y = f (y, x)
\end{Verbatim}
\begin{Verbatim}[frame=single]
uncurry :: (a -> b -> c) -> (a, b) -> c
uncurry f (x, y) = f x y
\end{Verbatim}
\end{enumerate}

\bigskip

\section*{zadanie 13}
\begin{itemize}
    \item Funkcja phi Eulera:
\begin{verbatim}
phi :: Int -> Int
phi n = length [x | x <- [1..n - 1], gcd x n == 1]
\end{verbatim}
    tworzy tablicę liczb od 1 do n-1 i następnie filtruje te, które są względnie pierwsze z n. length zwraca długość tej tablicy, co można utożsamiać z mocą zbioru.

    \item Funkcja $\sum_{k \mid n} \phi(k)$:
\begin{verbatim}
phi2 :: Int -> Int
phi2 n = sum [phi x | x <- [1..n], n `mod` x == 0]
\end{verbatim}
    tworzy tablicę liczb od 1 do n, filtruje te, które są dzielnikami liczby n i liczy sumę funkcji phi dla tych liczb. Zauważmy, że $\sum_{k \mid n} \phi(k) = n$, ponieważ każdą liczbę można zapisać jako sumę liczb względnie pierwszych w jej dzielnikach.
\end{itemize}

\bigskip

\section*{zadanie 14}
Liczba doskonała: $n = \sum \{d: 1 \leq d < n,\ d\mid n\}$

Na początku zdefiniujmy funkcję, która sprawdza czy dana liczba jest doskonała:
\begin{verbatim}
isPerfect :: Int -> Bool
isPerfect n = n == sum [k | k <- [1..n-1], n `mod` k == 0]
\end{verbatim}

Następnie zdefiniujmy funkcję, która zwróci wszystkie liczby doskonałe mniejsze od n:
\begin{verbatim}
allPerfect :: Int -> [Int]
allPerfect n = [k | k <- [1..n], isPerfect k]
\end{verbatim}

Dla n = 10000 otrzymamy: [6, 28, 496, 8128]

\bigskip

\section*{zadanie 15}
Na początku zdefiniujmy funkcję, która zwraca sumę dzielników:
\begin{verbatim}
sumOfDivisors :: Int -> Int
sumOfDivisors n = sum [k | k <- [1..n-1], n `mod` k == 0]
\end{verbatim}

Następnie zdefiniujmy funkcję, która sprawdza, czy 2 liczby są zaprzyjaźnione:
\begin{verbatim}
areSociable :: Int -> Int -> Bool
areSociable a b = sumOfDivisors a == b && sumOfDivisors b == a && a /= b
\end{verbatim}

Na koniec, zdefiniujmy funkcję, która zwróci wszystkie pary liczb zaprzyjaźnionych, mniejsze od podanego limitu:
\begin{verbatim}
socialPairs :: Int -> [(Int, Int)]
socialPairs limit =
    [(a, b) | a <- [1..limit],
              let b = sumOfDivisors a,
              areSociable a b && a < b && b < limit]
\end{verbatim}

Dla limit = $10^5$ otrzymamy:
[(220,284),(1184,1210),(2620,2924),(5020,5564),(6232,6368),(10744,10856), \par (12285,14595),(17296,18416)...]

\bigskip
\section*{zadanie 16}
Definiujemy funkcje: $dcp(n) = \frac{1}{n^2} |\{(k, l)\ \in\ \{1, 2, ..., n\}:\ gcd(k, l) = 1\}|$
\begin{itemize}
    \item Implementacja funkcji za pomocą list comprehension: \par
    \begin{verbatim}
    dcp1 :: Int -> Double
    dcp1 n = fromIntegral a / fromIntegral b
        where a = length [(k, l) | k <- [1..n], l <- [1..n], gcd k l == 1]
              b = n^2
    \end{verbatim}

    \item Implementacja funkcji rekurencyjnej
    \begin{verbatim}
        dcp' :: Int -> Double
        dcp' n = fromIntegral (countCoprimes n n 1 1) / fromIntegral (n ^ 2)
        
        countCoprimes :: Int -> Int -> Int -> Int -> Int
        countCoprimes n m i j
            | i > m = 0 --koniec
            | j > m = countCoprimes n m (i + 1) 1 --przejscie do nowego weirsza
            | gcd i j == 1 = 1 + countCoprimes n m i (j + 1) --liczmy wzglednie pierwsza pare
            | otherwise = countCoprimes n m i (j + 1) --nie jest wzglednie pierwsza
    \end{verbatim}

    \item $lim_{n \to \infty} dcp(n)$ \par
    kolejne wartości: \par
    \begin{verbatim}
        [0.6087,0.611575,0.6088333333333333,0.60846875,0.608924,
        0.6083305555555556,0.608234693877551,
        0.6085921875,0.6082111111111111,0.608383,0.6084586776859504,
        0.6080354166666667,0.6080988165680473,
        0.6082525510204082,0.6081613333333333,0.607993359375,0.6083678200692042,
        0.6080601851851852,0.6080096952908587,
        0.60829375,0.6080823129251701,0.6079518595041322,0.6081570888468809,
        0.6081019097222222,0.6079608,0.6081087278106508]
    \end{verbatim}
    Widzimy, że wraz ze wzrostem n, wartość dcp(n) zbiega do $\approx 0.608$. \par
    Możemy wysnuć hipotezę, że $lim_{n \to \infty} dcp(n) = \frac{6}{\pi^2}$.
\end{itemize}
\end{document}
