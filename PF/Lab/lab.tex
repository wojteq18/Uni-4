\documentclass{article}
\usepackage[utf8]{inputenc}
\usepackage[T1]{fontenc}
\usepackage{lmodern}
\usepackage[polish]{babel}
\usepackage{amsmath}
\usepackage{tikz}
\usepackage{algorithm}
\usepackage{algpseudocode}
\usepackage{hyperref}
\usepackage{float}
\usepackage{graphicx}



\title{Programowanie funkcyjne - laboratoria}
\author{Wojciech Typer}
\date{}

\begin{document}
\maketitle
\vspace{1\baselineskip}
\textbf{zadanie 1}
\vspace{1\baselineskip}
power x y = power $y^x$ \par
p2 = power 4 $\rightarrow$ power 4 y = $y^4$ \par
p3 = power 3 \par
(p2 . p3) 2 = p2(p32) = p 2 8 = $8^4$ = 4096 \par
p2 :: Int -> Int \par
p3 :: Int -> Int \par
(p2 . p3) :: Int -> Int \par
Wyrażenia lambda: \par
power = $/x \rightarrow /y \rightarrow y ^ x$ \par
p2 = $/y \rightarrow y ^ 4$ \par
p3 = $/y \rightarrow y ^ 3$ \par
\vspace{1\baselineskip}
\textbf{zadanie 2}
\end{document}