\documentclass{article}
\usepackage[utf8]{inputenc}
\usepackage[T1]{fontenc}
\usepackage{lmodern}
\usepackage[polish]{babel}
\usepackage{amsmath}
\usepackage{tikz}
\usepackage{algorithm}
\usepackage{algpseudocode}
\usepackage{hyperref}
\usepackage{float}
\usepackage{graphicx}
\usepackage{amsfonts}
\usepackage{fancyvrb}

\title{Programowanie funkcyjne}
\author{Wojciech Typer}
\date{}

\begin{document}

\maketitle

W katalogu przykłady pojawiać się będą przykłady z wykładów. 

\vspace{1\baselineskip}

\textbf{Funkcja lambda} 

\[
\text{exp} = (\lambda a: \text{Typ} \rightarrow (a \rightarrow a))
\]

\[
\text{exp} (\text{Int}) :: \text{Int} \rightarrow \text{Int}
\]

\[
\text{exp} (\text{Double}) :: \text{Double} \rightarrow \text{Double}
\]

\vspace{1\baselineskip}

\textbf{Matematyczny zapis inkrementacji} 

\[
\text{inc } x = x + 1
\]

\[
>:t \quad \text{inc}
\]

\[
\text{inc} :: \text{hum}(a) \implies (a \rightarrow a)
\]

\[
\text{exp} = (\forall a: \text{hum} \rightarrow (a \rightarrow a))
\]

\[
\text{exp} (\text{Int}) :: \text{Int} \rightarrow \text{Int}
\]

\[
\text{exp} (\text{Bool}) \leftarrow \text{(błąd)}
\]

\vspace{15\baselineskip}

\textbf{Typy w Haskellu}

\vspace{1\baselineskip}

\begin{itemize}
    \item Typy proste:
    \begin{itemize}
        \item \texttt{Int}
        \item \texttt{Double}
        \item \texttt{Char}
        \item \texttt{Bool}
    \end{itemize}
    
    \item Typy złożone:
    \begin{itemize}
        \item Listy
        \item Krotki
        \item Funkcje
    \end{itemize}
\end{itemize}
\vspace{1\baselineskip}
\textbf{Funkcja collatz'a} \par
\vspace{1\baselineskip}
Funkcja collatz'a jest nierozstrzygnięty dotychczas problem o wyjątkowo  \par
prostym jak wiele innych problemów teorii liczb sformułowaniu.
\vspace{1\baselineskip}

\[
c_{n+1} =
\begin{cases} 
\frac{1}{2} c_n & \text{gdy } c_n \text{ jest parzysta} \\
3c_n + 1 & \text{gdy } c_n \text{ jest nieparzysta}
\end{cases}
\]

lub

\[
c_{n+1} = \frac{1}{2} c_n - \frac{1}{4} (5c_n + 2) ((-1)^{c_n} - 1).
\]
\vspace{2\baselineskip}
\textbf{przykłady/W2 $\rightarrow$ funkcja collatz'a} \par
\vspace{1\baselineskip}
\textbf{Listy w Haskellu}

\vspace{1\baselineskip}

Definicja list w Haskellu:
\[
[a] = \text{lista elementów } a
\]

\[
= \{[a_{1}, ..., a_{k}]: a_{1}, ..., a_{k}, k \in \mathbb{N} \}
\]

\vspace{1\baselineskip}

\begin{figure}[H]
    \centering
    \includegraphics[width=0.8\textwidth]{/home/wojteq18/Pobrane/zdjecia/llist.png}
    \caption{Przykładowa lista w Haskellu}
    \label{fig:example_image}
\end{figure}

\vspace{1\baselineskip}

Struktura listy:
\[
x_0 : [x_1, x_2, ..., x_k] = [x_0, x_1, ..., x_k]
\]

\vspace{1\baselineskip}

Lista pusta:
\[
[] \quad \text{(lista pusta)}
\]

\vspace{1\baselineskip}

Przykłady list:
\[
[1, 2, 3] \quad \text{można zapisać jako} \quad 1:2:3:[]
\]

\vspace{1\baselineskip}

Dodawanie (konkatenacja) dwóch list:
\[
[1, 2, 3] + [4, 5] = [1, 2, 3, 4, 5]
\]

\vspace{1\baselineskip}
\textbf{Prelude} \par
\vspace{1\baselineskip}
Prelude to standardowa biblioteka Haskella \par
\vspace{1\baselineskip}
\begin{itemize}
    \item Dodawanie elementu na początku listy
        \begin{Verbatim}[frame=single]
>:t (1:[2,3])
(1:[2,3]) :: Num a => [a]
        \end{Verbatim}
    \item Konkatenacja list
        \begin{Verbatim}[frame=single]
>:t [1,2]++[3,4]
[1,2]++[3,4] :: Num a => [a]
        \end{Verbatim}
\end{itemize}

\subsubsection{Podstawowe funkcje operujące na listach}
\begin{itemize}
    \item \texttt{length :: [a] \(\to\) Int}
        \begin{itemize}
            \item \texttt{length [] = 0}
            \item \texttt{length (x:xs) = 1 + length xs}
        \end{itemize}
    \item \texttt{head :: [a] \(\to\) a} \newline
        zwraca pierwszy element listy
        \begin{itemize}
            \item \texttt{head (x:xs) = x}
            \item \texttt{head [] = error "empty list"}
        \end{itemize}
    \item \texttt{tail :: [a] \(\to\) [a]} \newline
        zwraca listę bez pierwszego elementu
        \begin{itemize}
            \item \texttt{tail (x:xs) = xs}
            \item \texttt{tail [] = error "empty list"}
        \end{itemize}
    \item \texttt{last :: [a] \(\to\) a} \newline
        zwraca ostatni element listy
        \begin{itemize}
            \item \texttt{last [x] = x}
            \item \texttt{last (x:xs) = last xs}
            \item \texttt{last [] = error "empty list"}
        \end{itemize}
    \item \texttt{filter :: (a \(\to\) Bool) \(\to\) [a] \(\to\) [a]}
        \begin{itemize}
            \item \texttt{filter p [] = []}
            \item \texttt{filter p (x:xs) = if p x then x : filter p xs else filter p xs}
            \item \texttt{filter (\(\lambda n \to n > 0\)) [-1,2,-3,4] = [2,4]}
            \item \texttt{filter even [1..10] = [2,4,6,8,10]}
        \end{itemize}
        Jak zdefiniować funkcję \textbf{filter}:
        \begin{Verbatim}[frame=single]
filter p [] = []
filter p (x:xs)
    | p x = x : filter p xs
    | otherwise = filter p xs
        \end{Verbatim}
    \item \texttt{map :: (a \(\to\) b) \(\to\) [a] \(\to\) [b]} \newline
        zwraca listę, która powstaje poprzez zastosowanie funkcji do każdego elementu listy
        \begin{itemize}
            \item \texttt{map f [] = []}
            \item \texttt{map f (x:xs) = f x : map f xs}
            \item \texttt{map (\(\lambda n \to n*n\)) [1,2,3] = [1,4,9]}
            \item \texttt{map (\(\lambda n \to n^3\)) [1..10] = [1,8,27,64,125,216,343,512,729,1000]}
        \end{itemize}
        gdzie \textbf{[1..10]} to skrót od \textbf{[1,2,3,4,5,6,7,8,9,10]}.
\end{itemize}
\vspace{1\baselineskip}
\textbf{List comprehension} \par
\vspace{1\baselineskip}
$[f x_1 x_2 x_3 | x_1 \leftarrow xs, x_2 \leftarrow ys, x_3 \leftarrow zs]$ \par
$[f x_1 x_2 x_3 | x_1 \leftarrow xs, x_2 \leftarrow, x_1 < x_2, x_3 \leftarrow zs]$
\vspace{1\baselineskip}
\begin{figure}[H]
    \centering
    \includegraphics[width=0.8\textwidth]{/home/wojteq18/Pobrane/zdjecia/lc.png}
    \label{fig:example_image}
\end{figure}
\vspace{1\baselineskip}
\textbf{Przykład: } trójki pitagorejskie \par
\vspace{1\baselineskip}
$[(x, y, z) | z \leftarrow [1..100], y \leftarrow [1..z], x \leftarrow [1..y], x^2 + y^2 + z^2$, gcd x y == 1] \par


\end{document}
