\documentclass[12pt]{article}
\usepackage[utf8]{inputenc}
\usepackage[T1]{fontenc}
\usepackage{lmodern}
\usepackage[polish]{babel}
\usepackage{amsmath}
\usepackage{tikz}
\usepackage{algorithm}
\usepackage{algpseudocode}
\usepackage{hyperref}
\usepackage{float}
\usepackage{graphicx}
\usepackage{booktabs} 

\title{Technologie sieciowe - lista 1}
\author{Wojciech Typer}
\date{}

\begin{document}
\maketitle

\vspace{1\baselineskip}

\textbf{Opis programów}
\begin{itemize}
  \item \textbf{Ping} $\rightarrow$ program używany do sprawdzania dostępności hosta w sieci IP.
  Wysyła pakiety ICMP Echo Request do docelowego adresu IP i mierzy czas odpowiedzi.
  Przydatny jest do mierzenia opóźnień i sprawdza, czy host jest osiągalny.
  \item \textbf{Traceroute} $\rightarrow$ program używany do analizy trasy pakietu do docelowego hosta,
  wyświetlając kolejno przechodzone routery. Używa pakietów ICMP lub UDP z rosnącym TTL (time-to-live),
  aby ujawnić każdy węzeł pośredni.
  \item \textbf{Wireshark} $\rightarrow$ program służący do przechwytywania i analizy pakietów sieciowych.
  Umożliwia monitorowanie ruchu w czasie rzeczywistym, filtrowanie danych i diagnozowanie problemów
  sieciowych. W wersji terminalowej dostępny jako tshark.
\end{itemize}

\textbf{Pingowanie serwerów}
\begin{itemize}
  \item Serwer w Polsce - Politechnika Wrocławska
    \begin{figure}[H]
      \centering
      \includegraphics[width=1.0\textwidth]{/home/wojteq18/Obrazy/Screenshots/Screenshot From 2025-03-19 19-11-33.png}
      \label{fig:ping_politechnika}
    \end{figure}
    \vspace{1\baselineskip}

  \item Serwer google.com
    \begin{figure}[H]
      \centering
      \includegraphics[width=1.0\textwidth]{/home/wojteq18/Obrazy/Screenshots/Screenshot From 2025-03-19 19-17-10.png}
      \label{fig:ping_google}
    \end{figure}

  \item Serwer w Australii - sydney.edu.au
    \begin{figure}[H]
      \centering
      \includegraphics[width=1.0\textwidth]{/home/wojteq18/Obrazy/Screenshots/Screenshot From 2025-03-19 19-17-55.png}
      \label{fig:ping_sydney}
    \end{figure}

  \item Serwer w Czechach - cuni.cz
    \begin{figure}[H]
      \centering
      \includegraphics[width=1.0\textwidth]{/home/wojteq18/Obrazy/Screenshots/Screenshot From 2025-03-19 19-19-12.png}
      \label{fig:ping_cuni}
    \end{figure}
    \vspace{3\baselineskip}

  \item Serwer w Chinach - fudan.edu.cn
    \begin{figure}[H]
      \centering
      \includegraphics[width=1.0\textwidth]{/home/wojteq18/Obrazy/Screenshots/Screenshot From 2025-03-19 19-19-51.png}
      \label{fig:ping_fudan}
    \end{figure}

  \item Serwer w Japonii - www.kyoto-u.ac.jp
    \begin{figure}[H]
      \centering
      \includegraphics[width=1.0\textwidth]{/home/wojteq18/Obrazy/Screenshots/Screenshot From 2025-03-19 19-20-36.png}
      \label{fig:ping_kyoto}
    \end{figure}

  \item Serwer w Niemczech - www.hu-berlin.de
    \begin{figure}[H]
      \centering
      \includegraphics[width=1.0\textwidth]{/home/wojteq18/Obrazy/Screenshots/Screenshot From 2025-03-19 19-21-11.png}
      \label{fig:ping_hu_berlin}
    \end{figure}
\end{itemize}

\vspace{5\baselineskip}

\textbf{Obserwacje: }

\vspace{1\baselineskip}
\begin{itemize}
  \item Liczba przeskoków (hops)
    \begin{itemize}
      \item Najmniejsza liczba przeskoków (hops = 6) wystąpiła w przypadku serwera Google, co może sugerować, że Google ma serwery na terenie Polski (lub w bliskiej odległości od Wrocławia).
      \item Największa liczba przeskoków (hops = 35) wystąpiła w serwerach w Chinach, co sugeruje, że pakiet przeszedł przez wiele pośrednich routerów i prawdopodobnie przez chińską sieć zaporową ("Great Firewall").
      \item Serwery w sąsiednich krajach (Czechy, Niemcy) mają stosunkowo małą liczbę przeskoków, co jest zgodne z ich bliską geograficzną lokalizacją.
    \end{itemize}

  \item Opóźnienia (time) a odegłości geograficzne
    \begin{itemize}
      \item Najkrótsze czas odpowiedzi miał serwer w Polsce (Politechniki Wrocławskiej), co jest zgodne z bliską lokalizacją geograficzną.
      \item Najdłuższy czas odpowiedzi miał serwer chiński, co może być skutkiem restrykcji sieciowych w Chinach.
      \item Serwer w Australii miał stosunkowo długi czas odpowiedzi, co jest zgodne z dużą odległością geograficzną.
    \end{itemize}
\end{itemize}

\vspace{1\baselineskip}

\textbf{Maksymalny niefragmentowany pakiet} \newline

Standardowym MTU (Maximum Transmission Unit) dla sieci Ethernet \par wynosi 1500 bajtów.

\vspace{1\baselineskip}
\textbf{Pingowanie serwerów z dużymi pakietami}

\begin{itemize}
  \item Serwer google.com
    \begin{figure}[H]
      \centering
      \includegraphics[width=1.0\textwidth]{/home/wojteq18/Obrazy/Screenshots/Screenshot From 2025-03-19 20-03-54.png}
      \label{fig:large_google}
    \end{figure}

  \item Serwer w Czechach - cuni.cz
    \begin{figure}[H]
      \centering
      \includegraphics[width=1.0\textwidth]{/home/wojteq18/Obrazy/Screenshots/Screenshot From 2025-03-19 20-04-47.png}
      \label{fig:large_cuni}
    \end{figure}

  \item Serwer w Chinach - fudan.edu.cn
    \begin{figure}[H]
      \centering
      \includegraphics[width=1.0\textwidth]{/home/wojteq18/Obrazy/Screenshots/Screenshot From 2025-03-19 20-05-58.png}
      \label{fig:large_fudan}
    \end{figure}

  \item Serwer w Japonii - www.kyoto-u.ac.jp
    \begin{figure}[H]
      \centering
      \includegraphics[width=1.0\textwidth]{/home/wojteq18/Obrazy/Screenshots/Screenshot From 2025-03-19 20-06-38.png}
      \label{fig:large_kyoto}
    \end{figure}

  \item Serwer w Niemczech - www.hu-berlin.de
    \begin{figure}[H]
      \centering
      \includegraphics[width=1.0\textwidth]{/home/wojteq18/Obrazy/Screenshots/Screenshot From 2025-03-19 20-07-45.png}
      \label{fig:large_hu_berlin}
    \end{figure}
\end{itemize}

\vspace{1\baselineskip}
\textbf{Obserwacje:}

\vspace{1\baselineskip}
\begin{itemize}
  \item Problemy z dostarczeniem pakietów do niektórych serwerów: Na serwery Politechniki Wrocławskiej i Uniwerystety w Sydney nie udało
  się przesłać tak dużych pakietów, prawdopodobnie z powodu ograniczeń MTU (Maximum Transmission Unit) lub obecności firewalli.
  \item Czas odpowiedzi od serwerów, które otrzymały pakiet wzrósł, co może sugerować, że większe pakiety wymagają dłuższego czasu
  przetworzenia przez routery.
  \item Liczba przeskoków (hops) pozostała taka sama, co sugeruje, że jest niezależna od rozmiaru pakietów.
\end{itemize}
\vspace{1\baselineskip}
\textbf{Podsumowanie pingowania} \newline
\begin{table}[h]
  \centering
  \begin{tabular}{cccccc} % Ustawione 6 kolumn (c = wyśrodkowane)
      \toprule
      \textbf{\#} & \textbf{Odległość} & \textbf{Bites} & \textbf{Liczba skoków stamtąd} & \textbf{Liczba skoków tam} & \textbf{Opóźnienie ping'u} \\ 
      \midrule
      1  & Blisko         & 100  & 52 & 13 & 40ms   \\ 
      2  & Blisko         & 1400 & 52 & 13 & 62ms \\ 
      3  & Daleko         & 100  & 53 & 16 & 55ms  \\ 
      4  & Daleko         & 1400 & 51 & 16 & 78ms  \\ 
      5  & Bardzo daleko  & 100  & 51 & 22 & 70ms    \\ 
      6  & Bardzo daleko  & 1400 & 51 & 22 & 73ms    \\ 
      \bottomrule
  \end{tabular}
  \caption{Parametry sieciowe w zależności od odległości}
  \label{tab:network_parameters}
\end{table}

\vspace{1\baselineskip}
\textbf{Traceroute:}

\begin{figure}[H]
  \centering
  \includegraphics[width=1.0\textwidth]{/home/wojteq18/Obrazy/Screenshots/Screenshot From 2025-03-19 20-36-54.png}
  \label{fig:traceroute_china}
\end{figure}

Najdłuższa trasa, jaką udało mi się znaleźć za pomocą programu traceroute, dociera do Chin, na serwer uniwersytetu w Fudan. 
W niektórych punktach (np. 14, 15, 16) widzimy, że router nie odpowiedział na zapytanie, co może świadczyć o tym, że pakiet przeszedł przez sieci wirtualne
(sieć VPN lub firewall, który nie odpowiada na ICMP).

\begin{figure}[H]
  \centering
  \includegraphics[width=1.0\textwidth]{/home/wojteq18/Obrazy/Screenshots/Screenshot From 2025-03-19 20-46-23.png}
  \label{fig:traceroute_poland}
\end{figure}

Zgodnie z intuicją i wcześniejszymi wynikami, droga na serwer Politechniki Wrocławskiej zbadanej przez program Traceroute jest znacznie krótsza,
choć wciąż natrafia na routery, które nie odpowiadają na zapytania ICMP.

\vspace{1\baselineskip}
\textbf{Wireshark:}

Program WireShark (w wersji terminalowej tshark) pozwala na przesłanie \par wiadomości do innego urządzenia działającego w tej samej sieci:
\begin{itemize}
  \item na urządzeniu nadawcy:
  \begin{verbatim}
    > echo -n "hej :)" | nc -u 192.168.100.14 12345
  \end{verbatim}
  \item na urządzeniu odbiorcy:
  \begin{verbatim}
    > sudo tshark -i any -f "udp port 12345" -T fields -e data
  \end{verbatim}
\end{itemize}

Odbiorca otrzyma wiadomość w zapisie heksadecymalnym, co jest zgodne z protokołem UDP. 
W prosty sposób może ją jednak zdekodować do postaci tekstowej:
\begin{verbatim}
> echo -n "68656a203a29" | xxd -r -p
> hej :)
\end{verbatim}

\vspace{1\baselineskip}
\textbf{Wnioski: }
\begin{itemize}
  \item \textbf{Ping} umożliwia szybką weryfikację osiągalności danego hosta, a także pomiar opóźnienia dwukierunkowego (RTT).
  \item \textbf{Wielkość pakietu} w istotny sposób przekłada się na opóźnienia. Zbyt duże pakiety wymagają fragmentacji, co może wydłużać RTT lub powodować brak odpowiedzi, zwłaszcza jeśli fragmentacja jest zabroniona (flaga \texttt{DF}).
  \item \textbf{Największy niepofragmentowany pakiet} z reguły odpowiada wartości MTU dla danej ścieżki. W powyższych testach wyniósł on typowe 1500 bajtów (1472 + 28).
  \item \textbf{Traceroute} obrazuje poszczególne routery na trasie, co znacząco ułatwia diagnozowanie problemów związanych z routingiem lub identyfikowanie źródeł opóźnień.
  \item \textbf{Wireshark} pozwala na dogłębną obserwację przesyłanych pakietów (w tym procesu fragmentacji) i analizę szczegółów takich jak nagłówki czy flagi, co sprzyja precyzyjnemu diagnozowaniu usterek.
  \item Odległość geograficzna wpływa na zwiększenie liczby \textit{skoków} i czas RTT. Dla serwerów położonych daleko wartość opóźnienia i liczba węzłów pośrednich rosły znacznie bardziej niż dla serwerów bliższych.
  \item \textbf{Asymetryczność tras} występuje często na skutek różnych polityk routingu stosowanych przez operatorów – obserwowano to w kilku spośród przeprowadzonych testów.
\end{itemize}

Reasumując, narzędzia \textbf{ping}, \textbf{traceroute} i \textbf{Wireshark} \par stanowią spójny zestaw do diagnostyki sieci:
\begin{itemize}
  \item \textbf{ping} – zapewnia szybką weryfikację dostępności hosta i pomiar opóźnienia,
  \item \textbf{traceroute} – pozwala ustalić dokładną drogę pakietu i zidentyfikować ewentualne wąskie gardła,
  \item \textbf{Wireshark} – służy do szczegółowej analizy ruchu sieciowego, włączając w to fragmentację pakietów oraz informacje zawarte w nagłówkach.
\end{itemize}
%ZAINSTALOWAC SCAPY i wysłać wiadomosci do siebie
\end{document}
